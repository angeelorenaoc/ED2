\section*{Resumen}

En la presente práctica se toma como base el procesador de un solo ciclo, y posteriormente se procede a adecuarlo para implementar el procesador ARM Pipelined sin su respectiva unidad de Hazards, asimismo, en esta primera parte se implementa la codificación de la instrucción NOP, con el fin de que se pueda hacer uso de esta cuando en el cifrado de las instrucciones se encuentre una dependencia de los datos que aún no están listos. En la segunda parte, se construyó la unidad de Hazards del Pinelined que soluciona los problemas de dependencia de datos que posee el procesador, garantizando así, una mejora con respecto al procesador de un solo ciclo desarrollado en la práctica pasada. En este procesador también se podrán usar las instrucciones LSL, LSR, ROR, ASR, MOV, y BL, puesto que al ya estar implementadas solo fue necesario adaptarlas a la ejecución del Pinelined; también, se recrea la secuencia de los LEDs en la FPGA con base en la memoria de instrucciones de la anterior práctica, para respaldar el correcto funcionamiento del procesador completo.

\footnotesize\textbf{Palabras clave: ARM,  Ensamblador, Hazards,  Pinelined ,Procesador.} 

     

    

