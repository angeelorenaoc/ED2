\section*{Recomendaciones para la Escritura del Reporte}

Emplee una apropiada redacción para la escritura del reporte. Por favor haga un uso correcto de las reglas gramaticales. Sea preciso, claro y breve. \\

Emplee palabras que indiquen exactamente lo que se quiere decir, usando siempre un lenguaje sencillo. Evite oraciones muy largas, las cuales dificultan el entendimiento. Preferiblemente, escriba oraciones cortas y separadas por puntos, no por comas. Para expresar varias ideas, use una frase corta para cada una. \\

Preferiblemente haga uso de imágenes, figuras o esquemas. De esta manera, se da mayor claridad sobre la idea que se quiere expresar. Recuerde dar una explicación breve de cada una de ellas. \\

Errores frecuente que se deberían evitar:\\

\begin{itemize}
    \item Ideas mezcladas. Normalmente, el reporte tiene una serie de secciones claramente definidas: introducción, marco teórico, materiales y métodos, análisis de resultados y conclusiones. Cada sección responde a una pregunta concreta. Sin embargo, a veces terminamos, por ejemplo, escribiendo sobre un concepto en la sección de análisis de resultados o una conclusión en la sección de métodos, lo cual dificulta la comprensión del texto, además que terminamos, posiblemente, repitiendo ideas. La recomendación es realizar un esquema del reporte con las secciones correspondientes y establecer las ideas que se adaptan a cada una. 

    \item Errores ortográficos. Para evitar llevar al lector a interpretaciones incorrectas, se recomienda aprovechar la herramienta auto-corrector disponible en distintas aplicaciones.

    \item Incorrecto uso de signos de puntuación (coma, punto y coma, punto seguido, punto y aparte, entre otros). De igual manera que en el caso anterior, el incorrecto uso de los signos de puntuación puede llevar al lector a interpretaciones incorrectas. Por favor revise donde ubica sus signos de puntuación. 

    \item Incorrecto uso de tiempos verbales. Cuando esté redactando una frase, evite hacer cambios de tiempo en la misma (por ejemplo: comenzar la frase en pasado y terminarla futuro). Adicionalmente, se recomienda emplear los siguientes tiempos verbales de acuerdo con la sección:

    \begin{itemize}
        \item Resumen: presente o pasado.
        \item Introducción: presente.
        \item Procedimiento experimental y resultados: pasado.
        \item Resultados: pasado o presente.
        \item Conclusiones: presente.
    \end{itemize}

    \item Sin referencias a otros autores. Cuando haga uso de una idea o texto de otros autores, recuerde expresarlas con sus propias palabras. Recuerde además citar la fuente. Evite la copia textual de otros textos, a menos que sea estrictamente necesario, caso en el cual deberá encerrar el texto entre comillas e indicar la fuente.
\end{itemize}






